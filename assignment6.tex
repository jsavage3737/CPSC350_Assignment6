\documentclass[12pt]{article}
\usepackage{amsmath}
\usepackage{graphicx}
\usepackage{hyperref}
\usepackage[latin1]{inputenc}

\title{CPSC 350 Assignment 6: Sorting Algorithms}
\author{Jack Savage}

\begin{document}
\maketitle

For this assignment, we ran 4 sorting algorithms on an array of 2,138 floats specified
by a given text file. The results were as follows:
\begin{itemize}
\item Quick sort: 498 time units
\item Insertion sort: 4,350 time units
\item Merge sort: 512 time units
\item Bubble sort: 13,934 time units

\end{itemize}

This program was written in C++, which is the fastest of the languages I've learned at Chapman
I've learned so far. Time differences between algorithms would likely be more drastic if written in an interpreted language like Python. The runtimes of each algorithm make sense as quick sort and merge sort are $O(n*log (n))$ while insertion sort and bubble sort are $O(n^2)$ in the average case. It is interesting to note, however, the somewhat severe time difference between bubble and insertion sort. 

\end{document}
